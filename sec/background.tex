\section{Background and related work}
\label{sec:background}

\paragraph*{The Light Transport Equation} 
The light transport equation (LTE) \cite{kajiyaRENDERINGEQUATION1986} is a recursive integral over radiometric quantities, which governs our approximation of light transport.

\begin{equation}
  \begin{aligned}
  L_o(p, \omega_o) &= L_e(p, \omega_o)\\
  &+ \int_{\Omega} f(p, \omega_o, \omega_i) L_o(p, \omega_o, \omega_i) \cos\theta_i d\omega_i,
  \end{aligned}
\end{equation}

The equation states that the outgoing radiance at point $p$ in direction $\omega_o$ is equal to the emitted radiance at that point and direction plus the portion of the incident light on $p$ that gets reflected towards $\omega_o$.

\paragraph{The direct lighting integral} 

The LTE can be reformulated into an infinite sum where each term represents the contribution of a given path length \cite*[]{veachROBUSTMONTECARLO}. In doing so, we can separate the calculation of direct and indirect lighting. The term accounting for direct light is the one we are concerned with evaluating, and is given by the integral:

\begin{equation}
  \int_{\Omega} f(p, \omega_o, \omega_i) L_d(p, \omega_i) \cos\theta_i d\omega_i,
  \label{eq:directlighting}
\end{equation}

where $p$ is the \emph{shading point} we are evaluating direct light on, $L_d$ is the \emph{direct} incident radiance from direction $\omega_i$ to the shading point, and $f$ is the Bidirectional Scattering Function (BSDF) at the shading point. 

\paragraph*{Monte Carlo Integration}

This integral cannot be solved analytically in general, so an unbiased estimate is obtained with \emph{Monte Carlo integration}. To compute this estimate, we need to sample $N$ directions $\omega_i$ with a probability density function (PDF) $p$ and apply the Monte Carlo estimator \cite{L1996MCTAOFPBR}:

\begin{equation}
  \frac{1}{N} \sum_{i=1}^N \frac{f(p, \omega_o, \omega_i) L_d(p, \omega_i) \cos\theta_i}{p(\omega_i)}
\end{equation}

To reduce the variance of Monte Carlo, we can use \emph{importance sampling} to choose the directions $\omega_i$ from a probability density function (PDF) similar to the integrand. When deciding on a distribution to sample from, it is worth clarifying that we are \emph{not} after the distribution with the lowest variance for a given number of samples, but instead the one with the lowest variance for a given amount of \emph{execution time} \cite{shirleyMonteCarloTechniques1996}. Thus, an effective sampling strategy may be worse than an ineffective one if the latter can be sampled more efficiently.

Returning to direct lighting, \autoref{eq:directlighting} is usually reformulated into a sum over the contributions of each light source in the scene \cite{shirleyMonteCarloTechniques1996}, and each term is estimated separately. To estimate direct light from a single emitter, we sample directions according to the emitters area or solid angle \cite*[]{shirleyMonteCarloTechniques1996} and then combined with BSDF sampling using Multiple Importance Sampling (MIS) \cite*[]{veachROBUSTMONTECARLO}. However, this approach ignores the light source's visibility when drawing the samples, which can lead to high variance in scenes where the light is partially occluded.

\paragraph{Light portals}
\emph{Light portals} are artist-specified regions in the scene which indicate an opening to an environment light, such as a window in an interior scene. During rendering, the portal is then used to focus the environment light sampling to those directions visible through the portal. They are highly effective for accelerating convergence in many scenes and are implemented in several production renderers such as Cycles \cite{blenderfoundationLightSettingsBlender} and Renderman \cite{pixaranimationstudiosPxrPortalLight}.

The portal area is typically sampled uniformly, although more advanced portal sampling strategies exist \cite{ogakiGeneralizedLightPortals2020}\cite{bitterliPortalMaskedEnvironment2015}. Nonetheless, these sampling strategies are designed for sampling \emph{environment lights}, and several challenges and opportunities are presented when applying the same technique to area lights. Unlike area lights, environment lights extend across the whole scene, so any direction sampled through the portal will also be directed to the environment. For this reason, existing portal sampling strategies are unsuitable for area lights.