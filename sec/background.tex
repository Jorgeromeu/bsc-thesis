\section{Background and related work}
\label{sec:background}

\paragraph*{The Light Transport Equation} 
The light transport equation (LTE) \cite{kajiyaRENDERINGEQUATION1986} is a recursive integral over radiometric quantities (reference), which governs our approximation of light transport.

\begin{equation}
  \begin{aligned}
  L_o(p, \omega_o) &= L_e(p, \omega_o)\\
  &+ \int_{\Omega} f(p, \omega_o, \omega_i) L_o(p, \omega_o, \omega_i) \abs{\cos\theta_i} d\omega_i,
  \end{aligned}
\end{equation}

The equation states that the exitant radiance at point $p$ in direction $\omega_o$ is equal to the emitted radiance at that point and direction plus the portion of the incident light on $p$ that gets reflected towards $\omega_o$.

\paragraph{The direct lighting integral} 

Repeatedly unfolding the recursive call, the LTE can be reformulated into the path-integral formulation \cite*{veachROBUSTMONTECARLO} which separates the contributions of each path length into separate terms. This reformulation allows us to calculate direct and indirect independently. , The direct lighting term is the one we are concerned with evaluating in this paper, and it is given by the integral:

\begin{equation}
  \int_{\Omega} f(p, \omega_o, \omega_i) L_d(p, \omega_i) \abs{\cos\theta_i} d\omega_i,
\end{equation}

where $L_d$ is the \emph{direct} radiance from direction $\omega_i$, $f$ is the Bidirectional Scattering Function (BSDF) and the cosine term accounts for Lamberts law.

\paragraph*{Monte Carlo Integration}
This integral cannot be solved analytically in general, thus we use \emph{Monte Carlo integration} as a means to numerically estimate it. To compute this estimate, we need to sample $N$ directions $\omega_i$ and apply the Monte Carlo estimator \cite{L1996MCTAOFPBR}:

\begin{equation}
  \frac{1}{N} \sum_{i=1}^N \frac{f(p, \omega_o, \omega_i) L_d(p, \omega_r) \abs{\cos\theta_i}}{p(\omega_i)}
\end{equation}

To reduce variance, we can use \emph{importance sampling} to choose the directions $\omega_i$ from a probability density function PDF similar to the integrand. However, when deciding on a distribution to sample from, its clarifying that we are \emph{not} after the distribution providing the lowest variance for a given number of samples, but instead the distribution providing the lowest variance for a given amount of time \cite{shirleyMonteCarloTechniques1996}, thus an effective sampling strategy may be worse than an ineffective one if the latter can be sampled more efficiently.

The most common sampling technique is to choose a light to be sampled for based on a precomputed PDF over the light sources in the scene \cite{shirleyMonteCarloTechniques1996}, weighted based on one or more heuristics such as light power (emitter sampling citations), and then using multiple importance sampling \cite{veachROBUSTMONTECARLO} to draw samples from to the PDF of the BSDF and the light's solid angle, which are then combined according to the contribution of each sample. This approach, however, ignores the visibility of the light source when drawing the samples, which can lead to high variance in scenes where the light is partially occluded, producing many blocked shadow rays.

\paragraph{Light portals}
\emph{Light portals} are artist-specified regions in the scene which indicate an opening to an environment light, such as a window in an interior scene. During rendering, the portal is then used to focus the environment light sampling to only those directions visible through the portal. They are highly effective for accelerating convergence in many scenes and are implemented in several production renderers such as Cycles \cite{LightSettingsBlender} and Renderman \cite{PxrPortalLight}.

The portal area is typically sampled uniformly, although more advanced portal sampling strategies exist \cite{ogakiGeneralizedLightPortals2020}\cite{bitterliPortalMaskedEnvironment2015}. Nonetheless, these sampling strategies are designed for sampling \emph{environment lights}, and several challenges and opportunities are presented when applying the same technique to area lights. Unlike area lights, environment lights extend across the whole scene, so any direction sampled through the portal will also be directed to the environment. For this reason, existing portal sampling strategies are unsuitable for area lights.