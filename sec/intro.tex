\section{Introduction}
\label{sec:intro}

Physically-based Monte Carlo light transport algorithms create photorealistic imagery with applications ranging from videogames to architecture visualization. While they can deliver accurate illumination, their most significant limitation is their high computational demands. 

Direct lighting is often one of the most significant components, so most rendering algorithms compute direct illumination separately, allowing for more effective light sampling. Direct lighting is computed by explicitly casting shadow rays from the shading point to the light sources. The probability distribution function (PDF) from which the shadow ray directions are sampled can have a high impact on the convergence rate of the scene. For example, consider a scene with a lamp occluded by a lampshade, all of the shadow rays blocked by the lampshade will have no contribution to the final image and are thus wasted sampling effort.

One widely used technique for sampling direct lighting from environment lights is \emph{light portals}. These are artist-specified regions which indicate an opening to the environment light. This information is then used during rendering to prioritize sampling the directions that go through the portal. Our contribution is extending the idea of light portals to accelerate the sampling of partially occluded area lights by repurposing concepts from shadow volume calculations.

We begin by reviewing Monte Carlo techniques for direct lighting calculations and provide an overview of existing use cases and sampling strategies for light portals (\autoref{sec:background}). Next, we expand on the details of our method (\autoref{sec:method}) and its implementation into an existing path tracer (\autoref{sec:implementation}). Lastly, we evaluate our portal sampling strategy against light sampling (\autoref{sec:results}), and discuss the results (\autoref{sec:discussion}) before concluding and giving final remarks (\autoref{sec:conclusion}).
