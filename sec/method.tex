\section{Method}
\label{sec:method}

We now present our method. We begin by explaining how the portal is represented and sampled for a scene with a single emitter and a single portal. Next, we give an effective yet potentially slow sampling strategy and show how it can be made more efficient by repurposing ideas from shadow calculations. Lastly, we generalize the provided method to multiple portals and show how the portal data can be incorporated into a spatial emitter distribution.

\subsection{Sampling a single light portal}

Consider a simple scene containing a single light source covered by a lampshade (\autoref{fig:simplescene}). The core idea behind our method is to have the artist include the geometry of the opening to the light source as a "portal" in the scene description. During rendering, if the shading point is in front of the portal, instead of sampling the light, we sample the portal using the method explained in (sampling the portal, thinking in terms of shadows).

\begin{figure}[H]
  \centering
  \def\svgwidth{0.4\columnwidth}
  \input{img/svg_out/simplescene.pdf_tex}
  \caption{Simple scene with a single light and single portal. Light is sampled in the region behind the portal (yellow), and the  portal is sampled in the region in front of it (blue)}
  \label{fig:simplescene}
\end{figure}

\subsection{Ensuring an unbiased estimate}
\label{sec:maintain_unbiased}

By sampling the portal, we only consider the directions to the light that also go through the portal. If the portal is well-placed, the light is only visible through the portal if the shading point is in front of it. Thus this method remains unbiased. However, if the portal is slightly misplaced, there may be directions facing the light which do not go through the portal \autoref{fig:misplaced_portala}. Ignoring these directions makes the sampling technique biased.

There is one issue with this approach, for the estimate to be unbiased there needs to be a nonzero probability of sampling every light carrying distribution.
\begin{figure}[H]
  \centering
  \def\svgwidth{0.6\columnwidth}
  \input{img/svg_out/misplaced_portal.pdf_tex}
  \caption{Illustration of how, part of the integration domain is ignored when sampling directions from a misplaced portal. If samples are instead drawn from a linear combination of the solid angles of the light and portal we ensure the entire integration domain is covered.}
  \label{fig:misplaced_portala}
\end{figure}

To resolve this, when sampling a portal, instead of only drawing samples according to the portal's solid angle, we sample from a linear combination of the portal's PDF and the target light's solid angle, ensuring the full integration is covered with a nonzero probability. We use MIS with the single sample model, and the balance heuristic \cite*[]{veachROBUSTMONTECARLO} to weigh the linear combination optimally. We opt for the single sample model as, in most cases, sampling the light over the portal gives a higher variance. The probability of sampling the light instead of the portal is an artist-specified parameter, defaulting to a low value such as 0.1.

\subsection{Effectively sampling the portal}
\label{sec:proj-sampling}

In \autoref{sec:background}, we explained why uniformly sampling the portal area is not a suitable strategy when applying light portals to area lights. If we want to only sample those directions that go through the portal and face the light, then the region that needs to be sampled is the intersection of the solid angles subtended by the emitter and the portal.

This region can be computed by first perspective-projecting the light area, from the perspective of the shading point, onto the portal and then clipping the projected region to the portal bounds. We then sample the solid angle subtended by the clipped region (\autoref{fig:projection}).

\begin{figure}[H]
  \centering
  \def\svgwidth{0.65\columnwidth}
  \input{img/svg_out/projection.pdf_tex}
  \caption{Projecion sampling. The light (yellow) is projected onto the portal-plane (blue) and clipped to the portal bounds. Samples are then drawn from the clipped region (green).}
  \label{fig:projection}
\end{figure} 

This sampling strategy is effective per sample but can be considerably slower than light or uniform portal sampling, introducing runtime overhead. In the next section, we show how some additional geometric calculations allow us to restrict the use of this sampling strategy to points where it provides the most significant advantage over more efficient alternatives.

\subsection{Thinking in terms of shadows}
\label{sec:antishadow}

An interesting way to think about light portals is to consider the shadow that the portal would cast if we were to replace it with an occluder. The area in front of the occluder can then be separated into four regions. These regions are called the \emph{umbra}, \emph{penumbra} \emph{antumbra} and the \emph{region out of shadow} \cite{hasenfratzSurveyRealtimeSoft2003}. 

These regions are relevant because the portal represents the precise opposite of an occluder, an \emph{opening} to the light. We call the imaginary shadow volume cast by the portal the \emph{antishadow}. The regions of the antishadow then take on a new meaning when we treat them from the portal's perspective. The portal-equivalent regions are shown in figure \ref*{sec:antishadow} and represent:

\begin{itemize}
  \item The \textbf{antiumbra} is the region where all direct paths to the light go through the portal.
  \item The \textbf{antipenumbra} is the region where some but not all paths to the light go through the portal.
  \item The \textbf{antiantumbra} is the region where the portal is directly in front of the light but does not fully cover it.
  \item The region \textbf{outside of the antishadow} is the area where none of the paths from the shading point to the light to the light pass through the portal
\end{itemize}

\begin{figure}[H]
  \centering
  \subfloat[2D representation of shadow and antishadow regions.]{
  \def\svgwidth{0.8\columnwidth}
  \input{img/svg_out/antishadow.pdf_tex}
  }
  \\
  \subfloat[View of the light and portal from the point of view of the shading point, in different regions of the portal antishadow.]{
  \def\svgwidth{0.7\columnwidth}
  \input{img/svg_out/antishadow-pov.pdf_tex}
  }
  \label{fig:antishadow}
  \caption{Two-dimensional representation of the regions of a shadow. If we replace the occluder with a light portal, the same geometric volumes take different roles.}
\end{figure}

Returning to our problem of sampling the light going through a portal, having an idea of which region of the antishadow the shading point is in can be used to choose the optimal sampling strategy. If the shading point is \textbf{outside of the antishadow}, then there is no light carrying directions through the portal to the light, so there is no need to sample.

If the shading point is in the portals \textbf{antiumbra}, then all direct paths from the shading point to the light go through the portal. Thus, the optimal sampling strategy is to ignore the portal and sample the light source itself. Conversely, when the shading point is in the \textbf{antiantumbra}, all directions to the portal also face the light. Hence, the optimal strategy is to ignore the light and sample the portal uniformly. Figure (antishadow b) gives a good intuition on why this works.

Lastly, if the shading point is in the \textbf{antipenumbra} then some but not all directions to the light go through the portal. In this case, we resort to using the effective, but slow projection sampling method explained in \autoref{sec:proj-sampling}. 

Returning to the simple scene from figure \autoref{fig:simplescene}, we can now partition the volume in front of the portal further by including the antishadow volumes. 

\begin{figure}[H]
  \def\svgwidth{0.5\columnwidth}
  \input{img/svg_out/simplescene-antishadow.pdf_tex}
  \label{fig:simplescene-antishadow}
  \caption{Simple scene employing antishadow volumes to weight the choice of sampling strategy depending on the location of the shading point.}
\end{figure}

By doing this, we restrict the use of the slow projection sampling technique to only the points in the antipenumbra and use faster strategies when there is no benefit to performing the projection.

\subsection{Computing antishadow volumes}

In \autoref{sec:antishadow}, we explained the concept of a light portal "antishadow" and how information about which region of the antishadow the shading point is in can be used to sample the portal efficiently. This section discusses how these antishadow volumes are computed and used to weigh sapling strategies.

For arbitrary polygonal objects, the shape of the umbra and penumbra regions can be embedded in a discontinuity mesh constructed from the edges and vertices of the light, and the portal \cite*{drettakisFastShadowAlgorithm1994}. However, for simplicity, we implement a version of this algorithm limited to rectangular aligned light and portal.

\begin{figure}[H]
  \centering
  \subfloat[Antiumbra]{
  \def\svgscale{0.24}
  \input{img/svg_out/aa-umbra.pdf_tex}
  \label{fig:umbra-calculation}
  }
  \subfloat[Antipenumbra]{
  \def\svgscale{0.24}
  \input{img/svg_out/aa-penumbra.pdf_tex}
  \label{fig:penumbra-calculation}
  }
  \caption{Computing the antiumbra and antipenumbra for aligned rectangles. The antiantumbra can be derived from the antiumbra.}
\end{figure}

The antiumbra is the frustum produced by connecting the adjacent vertices from the portal and the light (\autoref{fig:umbra-calculation}). Likewise, the antipenumbra is the frustum produced by connecting the \emph{opposite} vertices from the light to the portal (\autoref{fig:penumbra-calculation}). The antiantumbra does not need to be explicitly computed, as it is defined by the same planes that make up the antiumbra, but with their normals inverted.

\subsection{Multiple portals per light}
To generalize the method presented above to multiple lights per portal, we separately include each portal in the emitter distribution. This way, if a spatial emitter distribution is used we can prioritize portals facing the shading point. To maintain an unbiased estimate, one minor adjustment needs to be made. By including each portal separately, the contribution of the light needs to be scaled down accordingly. Instead of dividing by the probability of the portal being selected, we divide by the target emitter being selected.